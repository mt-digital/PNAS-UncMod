\documentclass[letterpaper,11.5pt]{scrartcl}
\usepackage{totcount}
% \documentclass[11pt]{report}
% \documentclass{report}
% \documentclass{book}
\usepackage[bookmarks, hidelinks]{hyperref}
\usepackage[chatter]{rotating}
\usepackage{amssymb,amsmath}
\usepackage[title]{appendix}
% \usepackage{fullpage}
\usepackage{tabulary}
\usepackage{tabularx}
\usepackage{float}
% \usepackage[margin=0.50in]{geometry}
\usepackage[margin=1.00in]{geometry}

\usepackage{booktabs}
\usepackage{pslatex}
\usepackage{apacite}
\usepackage{caption}
\usepackage{subcaption}
\usepackage{pgfplots}
\usepackage{wrapfig}
\usepackage[english]{babel}
\usepackage{lmodern}
\usepackage{setspace}
\doublespace
% \usepackage{url}
\usepackage{bigfoot}
\usepackage[export]{adjustbox}
\setlength\intextsep{0pt}

% Colored comments 
\usepackage{color}
\definecolor{myorange}{RGB}{240, 96, 0}
\newcommand{\mt}[1]{{\textcolor{myorange} {({\tiny MT:} #1)}}}

\definecolor{myblue}{RGB}{30,144,255}
\newcommand{\jhj}[1]{{\textcolor{myblue} {({\tiny JHJ:} #1)}}}

\definecolor{mypurple}{RGB}{148,0,211}
\newcommand{\cm}[1]{{\textcolor{mypurple} {({\tiny CM:} #1)}}}

\definecolor{mygreen}{RGB}{26, 153, 51}
\newcommand{\ps}[1]{{\textcolor{mygreen} {({\tiny PS:} #1)}}}

\newtotcounter{citnum} %From the package documentation
\def\oldbibitem{} \let\oldbibitem=\bibitem
\def\bibitem{\stepcounter{citnum}\oldbibitem}

\usepackage{graphicx}

\title{Uncertainty in social learning evolution}

\author{{}}

\begin{document}
\maketitle

\newcommand{\pisub}[1]{\pi_{\mathrm{#1}}}
\newcommand{\pilow}{\pisub{low}}
\newcommand{\pihigh}{\pisub{high}}
\newcommand{\piI}{\langle \pisub{I} \rangle}
\newcommand{\piS}{\langle \pisub{S} \rangle}

\newcommand{\meanvar}[1]{\langle #1 \rangle}
\newcommand{\meansl}{\meanvar{s}}
\newcommand{\meanpi}{\meanvar{\pi}}
\newcommand{\meansoc}{\meanvar{\pi_\mathrm{S}}}
\newcommand{\meanasoc}{\meanvar{\pi_\mathrm{A}}}
\newcommand{\meanT}{\meanvar{T}}

\begin{abstract}

Social learning is essential to survival. It is likely to evolve when it is more
efficient than asocial, trial-and-error learning. Theoretically, some uncertainty is
necessary for social learning to be necessary, but too much uncertainty makes
social information useless. This fact is empirically
supported across biology and human sciences. However, we lack a theoretical
framework to predict the effects of specific classes and types of uncertainty on social
learning evolution. Furthermore, existing models and experimental operationalizations of
uncertainty are ambiguously related, and 
tend to only consider a small number of sources of uncertainty and behavioral
choices.  Here we use evolutionary agent-based modeling to consolidate
models of uncertainty in social learning evolution to improve on these
shortcomings. We model a time-varying environment with a varying number
of possible behaviors agents could perform to acquire payoffs.
We show that environmental variability, ambiguous payoffs, 
larger decision sets, and shorter agent lifespans interact to 
affect social learning evolution in complex ways that 
sometimes confuse evolution about which is the optimal strategy, i.e., 
emergent evolutionary uncertainty. 
Our work strengthens social evolution theory that could help 
guide human, non-human, and artificial life towards optimal responses to existential
threats and new opportunities.\footnote{This document contains
\total{citnum} references.}  
\end{abstract}

\section{Introduction}

Social learning is essential to human and other species' everyday life and survival.
It allows individuals to solve problems when acquiring information from others is
more efficient than learning on one's own~\cite{Laland2004}. Theory predicts that
social learning should be favored in contexts with greater
uncertainty~\cite{BoydRicherson1985,Henrich1998}, up to a
point~\cite{Rogers1988,Feldman1996}; this prediction has received some empirical
support across species~\cite{McElreath2005,Kendal2018,Allen2019}.  However, the
meaning of the term \emph{uncertainty} often conflates environmental variability,
spatial heterogeneity, ambiguity or uncertainty about payoff structure, and other
variability that could be plausibly interpreted as \emph{uncertainty}.  Many models
of social learning evolution do not account for 
individual-level cognition~\cite{Heyes2016}, though humans
clearly have evolved cognitive mechanisms for dealing with 
uncertainty generally~\cite{Gershman2019,Schulz2019}.
Furthermore, social learning evolution theory tends to be organized around an
increasing number of modeling choices~\cite[Figure 1]{Kendal2018} instead
of around broader contextual assumptions such as uncertainty conditions. It remains
an outstanding question, then, to identify how different contexts and forms of
uncertainty might combine to affect social learning evolution in complex ways.
Answering this question is critical for understanding human behavior adapting to a
rapidly changing world, both to mitigate existential
threats~\cite{Moya2020,Jones2021}, and to captialize on new opportunities such as
transitioning to clean energy use and
production~\cite{NatureEnergyEditorialPromisesPremises2018,Brisbois2022}.

To gain a more systematic understanding of uncertainty and social learning evoluiton,
we identified a set of common ecological
uncertainty classes (``eigenvectors'' of uncertainty) used as factors
in social learning evolution models: (1) environmental variability; (2) payoff
ambiguity (difference between best behavior choice and others); (3) 
selection set size (number of ecologically possible behaviors); and 
(4) the lifespan of agents. These were chosen because modeling and empirical 
study designs across taxa tend use one or more of these uncertainty dimensions.
Environmental variability is sometimes modeled as catastrophic and 
happens only once in an individual's lifetime~\cite{Rogers1988}, or may be
modeled as occurring with a fixed or variable
probability~\cite{Feldman1996,McElreath2005}---in all these examples environments
can only be in one of two states. Payoff ambiguity is sometimes in the form
of differences in expected payoffs arising from different
behaviors or strategies~\cite{Enquist2007,Rendell2010}
or by varying the standard deviation of payoffs~\cite{McElreath2005}. 
Selection set size is often limited to one size throughout social learning
evolution studies: empirical studies of \emph{bombus terresteris} (bumble
bee)~\cite{Baracchi2018} and \emph{parus major} (great tit)~\cite{Aplin2017}
populations each provided two behaviors for
the bees and birds to perform, each yielding greater or smaller
payoffs depending on experimental treatment; studies in humans have used two
or three possible behaviors~\cite{McElreath2005,Toyokawa2019}; m
odeling studies have used either two 
behaviors~\cite{Feldman1996,Rendell2010} or four behaviors~\cite{Enquist2007} 
in understanding the role of uncertainty. The age or lifespan of individuals was
found to have an effect on social learning in both \emph{parus
major}~\cite{Aplin2017} and \emph{poecilia reticulata} 
(guppies)~\cite{Leris2016} \mt{Need better connection between lifespan and 
age, or better references to support use of lifespan}.
Shorter lifespans mean individuals
die more uncertain of which behaviors are optimal, and pass that uncertainty on
to their offspring. To answer our research questions about the combined effect
of various forms of uncertainty on social learning evolution, we developed an
agent-based model that incorporates all four of
these uncertainty variables. We systematically vary these parameters
to understand and predict their effects on social learning evolution.

\paragraph{Social learning evolution} (CRISTINA, PAUL)
To add to confusion about how to integrate diverse models of social learning
evolution, different models select different social learning components in
seemingly \emph{ad hoc} ways 
\begin{itemize}
  \item 
    Vertical, oblique, horizontal transmission
  \item
    Conformity, payoff-biased, frequency-dependent, etc. A note about how
    ``conformity'' is often not distinguished from other forms of social learning
  \item
    Review human studies
  \item
    Review non-human studies~\cite{Leris2016,Aplin2017,Avargues-Weber2018,Baracchi2018}
  \item
    More things from which we selected the operation of our model
\end{itemize}

\paragraph{Meaning of uncertainty in ecology and cognitive and social sciences} (JAMIE, PAUL)
We need to support our claim that these four classes of uncertainty really 
count as forms of uncertainty and show how it is consistent across ecology and
cognitive and social science.

\paragraph{Research overview}

\section{Model}

\paragraph{Model overview (introduce uncertainty variables)}

We developed an agent-based model of a society of $N$ individuals who each must decide
which of $B$ behaviors to perform at each time step. Each behavior is a ``bandit'',
a common modelling and experimental approach for representing behaviors with
probabilistic payoffs~\cite{SuttonBartoBook,McElreath2005,Rendell2010,Schulz2019}. 
Each behavior $b$ yields Bernoulli payoffs: payoff of 1 with probability $\pi_b$ and 
zero payoff with probability $1 - \pi_b$; 
$\pi_b$ is therefore also the expected payoff of behavior $b$. Agents must
decide which behavior to perform at each time step. To do this, agents 
use an explore-exploit strategy to sometimes try the most profitable behavior
they know about, and other times try alternatives that may pay off more reliably. 


\paragraph{Agents}

\paragraph{Dynamics: initialization, behaviors and payoffs, evolution}

\paragraph{Computational analyses and outcome measures}

\paragraph{Analysis 1: patterns in social learning evolution over $u$}

\paragraph{Analysis 2: evolutionary uncertainty when $\meansoc \approx \meanasoc$}

\paragraph{Disc. 1: Foundation for studying ssortative social learning}~\cite{Katsnelson2014}.

\paragraph{Disc. 2: Applications to artificial
intelligence}~\cite{Sandholm1996,Jaques2019,Ndousse2021,Gronauer2022}

\bibliographystyle{apacite}
\bibliography{/Users/mt/workspace/Writing/library.bib}

\end{document}
